\PassOptionsToPackage{table}{xcolor}
\documentclass[aspectratio=169, usepdftitle=false, handout]{beamer}

\usepackage[english, secslide, subsecslide, coloraccent=darkmaroon]{beamerthemeawesome}

\addbibresource{refs.bib}

\usepackage{emoji}
\usepackage{graphicx}
\usepackage{textcomp}
\usepackage{pgfplots}
\usepackage{pgfplotstable}
\usepackage[
	duration=20,
	enduserslide=20,
	hidenotes
]{pdfpc}
\newcommand<>{\talknote}[1]{\only#2{\pdfpcnote{- #1}\relax}}

\usepackage{tikzlings-chickens}
\usepackage{tikzlings-koalas}
\usepackage{tikzducks}

\setminted{
	numbers=none,
	frame=none
}
\usemintedstyle{stata}

\title[SIMD]{SIMD}
\author[Lukas Pietzschmann]{Lukas Pietzschmann}
\subtitle{Single instruction, multiple data}
% \subtitle{Synchronized Intergalactic\\Marshmallow Dispensers}
\email{lukas.pietzschmann@uni-ulm.de}
\institute{Institute of Software Engineering and\\Programming Languages\\Institute of Distributed Systems}
\uni{University of Ulm}
\location{Ulm}
\date{07/04/2023}
\background{background.png}

\begin{document}

\maketitle

\section{Motivation}
\begin{frame}[fragile]
	\frametitle{How not to do it}
	\talknote{Sprungvorhersage ein mal falsch (25\%)}
	\talknote{Rufen wir das oft auf sind 25\% signifikant}
	\talknote{Wir hätten 4 oder 8 floats multiplizieren können}
	\begin{columns}[c]
		\begin{column}{0.45\textwidth}
			\begin{minted}{c}
				void mul4(float* arr) {
					for(int i=0; i < 4; ++i) {
						const float f = arr[i];
						arr[i] = f * f;
					}
				}
			\end{minted}
		\end{column}
		\begin{column}{0.45\textwidth}
			\begin{block}[Why is it that bad?]
				\begin{itemize}
					\item<2-> Short loops are bad
						\begin{itemize}
							\item Branch prediction will be wrong often
						\end{itemize}
					\item<3-> We could have multiplied way more than two floats
				\end{itemize}
			\end{block}
		\end{column}
	\end{columns}
\end{frame}

\begin{frame}[fragile]
	\frametitle{How to make it better}
	\talknote<1-5>{Keine Sprungvorhersage mehr}
	\talknote<1-5>{Komplettes Potential von out of order execution ausnutzen}
	\talknote<1-5>{Ladelatenzen verstecken}
	\talknote<1-5>{Leichter zu lesender maschinen-code}
	\talknote<6>{Klicken!}
	\begin{onlyenv}<1-5>
		\begin{columns}[c]
			\begin{column}{0.45\textwidth}
				\begin{minted}{c}
				void mul4(float* vec) {
					__m128 f=_mm_loadu_ps(vec);
					f = _mm_mul_ps(f, f);
					_mm_storeu_ps(vec, f);
				}
			\end{minted}
			\end{column}
			\begin{column}{0.45\textwidth}
				\begin{block}[Why is it better?]
					\begin{itemize}
						\item<2-> No loops
						\item<3-> No branches to predict
						\item<4-> Nice machine code
						\item<5-> We square all floats \enquote{at once}
					\end{itemize}
				\end{block}
			\end{column}
		\end{columns}
	\end{onlyenv}
	\mode<beamer>{
	\begin{onlyenv}<6>
		\hfill\huge\color{accent}\bfseries\href{https://godbolt.org/z/ET5qqerdj}{Or just compile with optimisations}\hfill
	\end{onlyenv}
	\begin{onlyenv}<7>
		\begin{columns}[c]
			\begin{column}{0.45\textwidth}
				\begin{minted}{c}
					void mul4(float* vec) {
						__m128 f=_mm_loadu_ps(vec);
						f = _mm_mul_ps(f, f);
						_mm_storeu_ps(vec, f);
					}
				\end{minted}
			\end{column}
			\begin{column}{0.50\textwidth}
				\setlength\fboxsep{1pt}
				\setlength\fboxrule{2pt}
				\begin{center}
					\textcolor{accent}{\fbox{\href{https://www.memedroid.com/memes/detail/1613336}{\includegraphics[width=.5\textwidth]{img/whatarethose.png}}}}
				\end{center}
				\begin{itemize}
					\item \mintinline{c}{__m128}
					\item \mintinline{c}{_mm_loadu_ps(float*)}
					\item \mintinline{c}{_mm_mul_ps(__m128, __m128)}
					\item \mintinline{c}{_mm_store_ps(float*, __m128)}
				\end{itemize}
			\end{column}
		\end{columns}
	\end{onlyenv}
}
\end{frame}

\mode<handout>
\begin{frame}
	\frametitle{How to make it better}
	\centering
	\huge\color{accent}\bfseries\href{https://godbolt.org/z/ET5qqerdj}{Or just compile with optimisations}
\end{frame}
\begin{frame}[fragile]
	\frametitle{How to make it better}
	\begin{columns}[c]
		\begin{column}{0.45\textwidth}
			\begin{minted}{c}
				void mul4(float* vec) {
					__m128 f=_mm_loadu_ps(vec);
					f = _mm_mul_ps(f, f);
					_mm_storeu_ps(vec, f);
				}
			\end{minted}
		\end{column}
		\begin{column}{0.50\textwidth}
			\setlength\fboxsep{1pt}
			\setlength\fboxrule{2pt}
			\begin{center}
				\textcolor{accent}{\fbox{\href{https://www.memedroid.com/memes/detail/1613336}{\includegraphics[width=.5\textwidth]{img/whatarethose.png}}}}
			\end{center}
			\begin{itemize}
				\item \mintinline{c}{__m128}
				\item \mintinline{c}{_mm_loadu_ps(float*)}
				\item \mintinline{c}{_mm_mul_ps(__m128, __m128)}
				\item \mintinline{c}{_mm_store_ps(float*, __m128)}
			\end{itemize}
		\end{column}
	\end{columns}
\end{frame}
\mode<all>

\begin{frame}
	\frametitle[\cite{const}]{Performance}
	\talknote{DPPS = Dot Product of Packed Single Precision Floating-Point Values}
	\talknote{VDPPS macht nur etwas mehr}
	\talknote{micro seconds}
	\begin{table}[]
		\begin{tabular}{l||c}
			\hline
			\rowcolor{accent!30}
			\textbf{Description}        & \textbf{Time (in \textmu s)} \\
			\hline\hline
			Regular floating point math & 439                          \\
			SSE dpps instruction        & 181                          \\
			AVX vdpps instruction       & 103
		\end{tabular}
		\caption{Time it takes to computes the dot product of two vectors with a length
			of 256,000}
	\end{table}
	\begin{columns}[c]
		\begin{column}{0.75\textwidth}
			\begin{itemize}
				\item<2-> [\triangleright] SSE: 2.5x speed increase
				\item<3-> [\triangleright] AVX: \phantom{2.}4x speed increase
			\end{itemize}
		\end{column}
		\begin{column}{0.15\textwidth}
			\begin{visibleenv}<2->
				\begin{tikzpicture}
					\duck[mask=red,cape=red,alien=green,laughing,xscale=-1, scale=.7]
				\end{tikzpicture}
			\end{visibleenv}
		\end{column}
	\end{columns}
\end{frame}

\section{Overview}
\begin{frame}
	\frametitle[\cite{flynn}]{Forms of computing systems}
	\talknote{SISD normal (eine CPU)}
	\talknote{MISD eher theoretisches konzept}
	\talknote{MIMD normal (mehrere CPUs)}
	\begin{table}[]
		\begin{tabular}{l||c|c}
			\hline
			\rowcolor{accent!30}
			                               & \textbf{Single data stream} & \textbf{Multiple data stream}                                                                         \\
			\hline\hline
			\textbf{Single instruction}    & SISD                        & \only<2>{\emoji{flying-saucer}}\textcolor<2>{accent}{\textbf<2>{SIMD}}\only<2>{\emoji{flying-saucer}} \\
			\textbf{Multiple Instructions} & MISD                        & MIMD
		\end{tabular}
	\end{table}
\end{frame}

% \begin{frame}
% 	\frametitle{History}
% 	% https://www.researchgate.net/publication/305762983_Performance_and_Scalability_Improvements_for_Discontinuous_Galerkin_Solutions_to_Conservation_Laws_on_Unstructured_Grids/link/5e6bb80092851c6ba7008da2/download
% 	\begin{itemize}
% 		\item 1993 P5
% 		\item 1995 P6
% 		\item 1997 MMX
% 		\item 1999 SSE
% 		\item 2000 SSE2
% 		\item 2004 SSE3
% 		\item 2006 SSSE3
% 		\item 2007 SSSE4.1
% 		\item 2008 SSSE4.2
% 		\item 2011 AVX
% 		\item 2013 AVX2
% 	\end{itemize}
% \end{frame}


\begin{frame}
	\frametitle{SIMD Support}
	\framesubtitle{According to Steam}
	\talknote{1993 P5}
	\talknote{1995 P6}
	\talknote{1997 MMX}
	\talknote{1999 SSE}
	\talknote{2000 SSE2}
	\talknote{2004 SSE3}
	\talknote{2006 SSSE3}
	\talknote{2007 SSSE4.1}
	\talknote{2008 SSSE4.2}
	\talknote{2011 AVX}
	\talknote{2013 AVX2}
	\centering
	\pgfplotsset{
		every axis legend/.append style={
				at={(1.02,0.5)},
				anchor=west,
				style={roundednode, fill=accent!3},
				legend cell align=left,
			},
	}
	\pgfplotsset{width=\textwidth, height=.7\textheight, compat=1.18}
	\pgfplotstableread[col sep=comma,]{assets/SSE.csv}\ssea
	\pgfplotstableread[col sep=comma,]{assets/SSE3.csv}\sseb
	\pgfplotstableread[col sep=comma,]{assets/SSE41.csv}\ssec
	\pgfplotstableread[col sep=comma,]{assets/SSE42.csv}\ssed
	\pgfplotstableread[col sep=comma,]{assets/SSE4A.csv}\ssee
	\pgfplotstableread[col sep=comma,]{assets/AVX.csv}\avxa
	\pgfplotstableread[col sep=comma,]{assets/AVX2.csv}\avxb
	\pgfplotstableread[col sep=comma,]{assets/SSSE3.csv}\ssse
	\href{https://raw.githubusercontent.com/jdegene/steamHWsurvey/master/shs.csv}{
		\begin{tikzpicture}
			\begin{axis}[axis background/.style={fill=accent!3},
				ylabel=Support (in \%), axis line style={rounded corners}, xtick=\empty,
				ytick=\empty, extra x ticks={0, 86, 173}, extra x tick
				labels={2008, 2015/16, 2023}]
				\addplot[color=accent, smooth] table [x=date, y=percentage, col sep=comma, x expr=\coordindex]{\ssea};
				\addlegendentry{SSE2}
				\addplot[color=blue, smooth] table [x=date, y=percentage, col sep=comma, x expr=\coordindex]{\sseb};
				\addlegendentry{SSE3}
				\addplot[color=green, smooth] table [x=date, y=percentage, col sep=comma, x expr=\coordindex]{\ssec};
				\addlegendentry{SSE4.1}
				\addplot[color=maroon, smooth] table [x=date, y=percentage, col sep=comma, x expr=\coordindex]{\ssed};
				\addlegendentry{SSE4.2}
				\addplot[color=red, smooth] table [x=date, y=percentage, col sep=comma, x expr=\coordindex]{\ssee};
				\addlegendentry{SSE4a}
				\addplot[color=yellow, smooth] table [x=date, y=percentage, col sep=comma, x expr=\coordindex]{\avxa};
				\addlegendentry{AVX}
				\addplot[color=orange, smooth] table [x=date, y=percentage, col sep=comma, x expr=\coordindex]{\avxb};
				\addlegendentry{AVX2}
				\addplot[color=olive, smooth] table [x=date, y=percentage, col sep=comma, x expr=\coordindex]{\ssse};
				\addlegendentry{SSSE3}
			\end{axis}
		\end{tikzpicture}}
	\let\thefootnote\relax\footnote{Note that Steam did not start collecting data for
		all extensions at the date of their release!}
	\addtocounter{footnote}{-1}
\end{frame}


\section{How to use SIMD}
\subsection{Overview}
\begin{frame}
	\centering
	\huge\color{accent}\bfseries It's hard
	\only<2>{\begin{tikzpicture}[overlay, remember picture]
			\node [draw=none, anchor=south east] at (current page.south east) {\includegraphics{img/whatshesaid.png}};
		\end{tikzpicture}}
\end{frame}

\begin{frame}
	\frametitle{The idea behind SIMD}
	\talknote{Erweiterung des Befehls und Registersatzes}
	\centering
	\begin{onlyenv}<all:-5>%
		Without SIMD:\\[7mm]
		\begin{tikzpicture}
			\node [squarenode] (A1) [] {A\textsubscript1};
			\node [squarenode] (A2) [below=of A1] {A\textsubscript2};
			\node [squarenode] (A3) [below=of A2] {A\textsubscript3};
			\node [squarenode] (A4) [below=of A3] {A\textsubscript4};
			\node [roundednode, fill=gray!10, node on layer=background] (F) [fit=(A1)(A2)(A3)(A4)] {};

			\node [squarenode, xshift=1.5cm] (B1) [right=of A1] {B\textsubscript1};
			\node [squarenode, xshift=1.5cm] (B2) [right=of A2] {B\textsubscript2};
			\node [squarenode, xshift=1.5cm] (B3) [right=of A3] {B\textsubscript3};
			\node [squarenode, xshift=1.5cm] (B4) [right=of A4] {B\textsubscript4};
			\node [roundednode, fill=gray!10, node on layer=background] (F2) [fit=(B1)(B2)(B3)(B4)] {};

			\node [squarenode, xshift=1.5cm, visible on=<all:2->] (C1) [right=of B1] {C\textsubscript1};
			\node [squarenode, xshift=1.5cm, visible on=<all:3->] (C2) [right=of B2] {C\textsubscript2};
			\node [squarenode, xshift=1.5cm, visible on=<all:4->] (C3) [right=of B3] {C\textsubscript3};
			\node [squarenode, xshift=1.5cm, visible on=<all:5->] (C4) [right=of B4] {C\textsubscript4};
			\node [roundednode, fill=gray!10, node on layer=background] (F3) [fit=(C1)(C2)(C3)(C4)] {};

			\node [draw=none] (O) at ($(F)!0.5!(F2)$) {$+$};
			\node [draw=none] (R) at ($(F2)!0.5!(F3)$) {$=$};

			\foreach \i in {1,...,4}{
					\draw [arrow, visible on=<all:\the\numexpr \i + 1\relax>, shorten >=3mm,shorten <=3mm] (A\i) to (B\i);
					\draw [arrow, visible on=<all:\the\numexpr \i + 1\relax>, shorten >=3mm,shorten <=3mm] (B\i) to (C\i);
				}
		\end{tikzpicture}
	\end{onlyenv}%
	\begin{onlyenv}<all:6->%
		With SIMD:\\[7mm]
		\begin{tikzpicture}
			\node [squarenode] (A1) [] {A\textsubscript1};
			\node [squarenode] (A2) [below=of A1] {A\textsubscript2};
			\node [squarenode] (A3) [below=of A2] {A\textsubscript3};
			\node [squarenode] (A4) [below=of A3] {A\textsubscript4};
			\node [roundednode, fill=gray!10, node on layer=background] (F) [fit=(A1)(A2)(A3)(A4)] {};

			\node [squarenode, xshift=1.5cm] (B1) [right=of A1] {B\textsubscript1};
			\node [squarenode, xshift=1.5cm] (B2) [right=of A2] {B\textsubscript2};
			\node [squarenode, xshift=1.5cm] (B3) [right=of A3] {B\textsubscript3};
			\node [squarenode, xshift=1.5cm] (B4) [right=of A4] {B\textsubscript4};
			\node [roundednode, fill=gray!10, node on layer=background] (F2) [fit=(B1)(B2)(B3)(B4)] {};

			\node [squarenode, xshift=1.5cm, visible on=<all:7>] (C1) [right=of B1] {C\textsubscript1};
			\node [squarenode, xshift=1.5cm, visible on=<all:7>] (C2) [right=of B2] {C\textsubscript2};
			\node [squarenode, xshift=1.5cm, visible on=<all:7>] (C3) [right=of B3] {C\textsubscript3};
			\node [squarenode, xshift=1.5cm, visible on=<all:7>] (C4) [right=of B4] {C\textsubscript4};
			\node [roundednode, fill=gray!10, node on layer=background] (F3) [fit=(C1)(C2)(C3)(C4)] {};

			\node [draw=none] (O) at ($(F)!0.5!(F2)$) {$+$};
			\node [draw=none] (R) at ($(F2)!0.5!(F3)$) {$=$};

			\foreach \i in {1,...,4}{
					\draw [arrow, visible on=<all:7>, shorten >=3mm,shorten <=3mm] (A\i) to (B\i);
					\draw [arrow, visible on=<all:7>, shorten >=3mm,shorten <=3mm] (B\i) to (C\i);
				}
		\end{tikzpicture}
	\end{onlyenv}
\end{frame}

\begin{frame}
	\frametitle{Approach}
	\talknote<1>{RAM ist langsam}
	\talknote<1>{Cash miss 100-200 Zyklen}
	\talknote<2>{2: Compiler kann Register Druck hoch halten}
	\begin{enumerate}
		\item Operate directly on memory
	\end{enumerate}
	\pause
	\vspace{2mm}
	\hspace{8.2mm}or
	\vspace{2mm}
	\begin{enumerate}
		\item Load data to registers
		\item Do as much as possible while it’s in registers
		\item Store results \ldots
		      \begin{itemize}
			      \item into memory
			      \item in general purpose registers
		      \end{itemize}
	\end{enumerate}
\end{frame}

\begin{frame}
	\frametitle{SIMD in C++}
	\framesubtitle{Intrinsics}
	\talknote{Compiler sieht eine normale Funktion die er normal optimieren kann}
	\talknote{Intrinsics bilden nicht immer auf nur eine Instruktion ab}
	\begin{block}[Intrinsics\hfill\color{white!70!accent}In\ldots\ what?]
		\begin{itemize}
			\item Usually implemented \enquote{inside} the compiler
			\item Allow for better optimisations than raw inline assembly
			\item Intrinsics provide access to instructions that cannot be generated
			      using the standard constructs
		\end{itemize}
	\end{block}
\end{frame}

\subsection{Data types}
\begin{frame}
	\frametitle{Register types}
	\talknote{xmm Register}
	\talknote{6 typen, für die CPU nur 2}
	\talknote{__m128 für register, float für RAM}
	\begin{columns}[c]
		\begin{column}{0\textwidth}
			\begin{tikzpicture}[overlay, remember picture]
				\node[inline, xshift=1cm, visible on=<2>] (A) {SSE2};
			\end{tikzpicture}
		\end{column}
		\begin{column}{\textwidth}
			\begin{table}[]
				\begin{tabular}{l||c|c}
					\hline
					\rowcolor{accent!30}
					                                                            & \textbf{16 Bytes} & \textbf{32 Bytes} \\
					\hline\hline
					\textbf{\phantom{32/}32 Bit float}                          & \texttt{__m128}   & \texttt{__m256}   \\
					\tikz\node[inline](B){\textbf{\phantom{32/}64 Bit double}}; & \texttt{__m128d}  & \texttt{__m256d}  \\
					\textbf{32/64 Bit integer}                                  & \texttt{__m128i}  & \texttt{__m256i}
				\end{tabular}
			\end{table}
		\end{column}
	\end{columns}
	\begin{tikzpicture}[overlay, remember picture]
		\draw[textarrow, shorten >=3mm,shorten <=3mm, visible on=<2>] (A) to (B.west);
	\end{tikzpicture}\par
	\onslide<3>{Note that:
		\begin{itemize}
			\item The CPU doesn't distinguish between \mintinline{c}{__m128}, \mintinline{c}{__m128d} and \mintinline{c}{__m128i}
			      \begin{itemize}
				      \item This information is only used for type checking
			      \end{itemize}
			\item The compiler automatically assigns the values to registers
			      \begin{itemize}
				      \item Be aware that there are only 16 (8+8) registers underneath the compiler
			      \end{itemize}
		\end{itemize}}
\end{frame}

\begin{frame}
	\frametitle{Register types}
	\framesubtitle{Gotcha!}
	\talknote{Für unser mental model, intern keinen unterschied}
	\talknote{Für Instruktionen die auf den niederwertigsten bits arbeiten egal}
	\begin{columns}[t]
		\begin{column}{0.45\textwidth}
			\centering
			\begin{tikzpicture}
				\node [roundednode, node on layer=foreground, fill=none, draw=none] (F1) {3.14159265358};
				\node [roundednode, fit=(F1), fill=white, node on layer=main, fill opacity=0.8] (A) {};
				\node [draw=none, anchor=south] at (A.north west) {0};
				\node [draw=none, anchor=south] at (A.north east) {127};
				\node [draw=none, anchor=center, node on layer=background, yshift=-4mm, visible on=<2->] at (A.center)
				{\color{red}\fontsize{150}{180}\selectfont\texttimes};
			\end{tikzpicture}\par
			\onslide<2->{A SIMD register does not store a single scalar value}
		\end{column}%
		\begin{column}{0.45\textwidth}%
			\centering
			\begin{tikzpicture}[node distance=2mm]
				\node [visible on=<3->, roundednode, node on layer=foreground, fill=none] (F1) {3.14};
				\node [visible on=<3->, roundednode, node on layer=foreground, fill=none] (F2) [right=of F1] {1.59};
				\node [visible on=<3->, roundednode, node on layer=foreground, fill=none] (F3) [right=of F2] {2.65};
				\node [visible on=<3->, roundednode, node on layer=foreground, fill=none] (F4) [right=of F3] {3.58};
				\node [visible on=<3->, roundednode, fit=(F1)(F4), fill=white, node on layer=main, fill opacity=0.8] (A) {};
				\node [visible on=<3->, draw=none, anchor=south] at (A.north west) {0};
				\node [visible on=<3->, draw=none, anchor=south] at (A.north east) {127};
				\node [visible on=<3->, draw=none, anchor=center, node on layer=background,yshift=-3.3mm, visible on=<4->] at (current bounding box.center)
				{\color{green}\fontsize{130}{156}\selectfont\checkmark};
			\end{tikzpicture}\par
			\onslide<4->{but multiple values that are interpreted like a vector.}
		\end{column}
	\end{columns}
\end{frame}

\subsection{Instructions}
% \begin{frame}
% 	\frametitle{Casting}
% 	\framesubtitle{Deutschland sucht die nächste Super-Instruktion}
% 	\begin{table}[]
% 		\small
% 		\begin{tabular}{l||c|c|c}
% 			\hline
% 			\rowcolor{accent!30}
% 			$\downarrow$~To / From~\rightarrow & \texttt{__m128}           & \texttt{__m128d}          & \texttt{__m128i}          \\
% 			\hline\hline
% 			\texttt{__m128}                    & \texttimes                & \texttt{_mm_castpd_ps}    & \texttt{_mm_castsi128_ps} \\
% 			\texttt{__m128d}                   & \texttt{_mm_castpd_pd}    & \texttimes                & \texttt{_mm_castsi128_pd} \\
% 			\texttt{__m128i}                   & \texttt{_mm_castps_si128} & \texttt{_mm_castpd_si128} & \texttimes
% 		\end{tabular}
% 	\end{table}
% 	\begin{itemize}
% 		\item Zero overhead (C++ devs love this stuff)
% 		      \begin{itemize}
% 			      \item [\triangleright] No \emph{actual} conversion
% 		      \end{itemize}
% 	\end{itemize}
% \end{frame}
%
% \begin{frame}
% 	\frametitle{Converting}
% 	\begin{table}[]
% 		\small
% 		\begin{tabular}{l||c|c|c}
% 			\hline
% 			\rowcolor{accent!30}
% 			$\downarrow$~To / From~\rightarrow & \texttt{__m128}          & \texttt{__m128d}         & \texttt{__m128i}         \\
% 			\hline\hline
% 			\texttt{__m128}                    & \texttimes               & \texttt{_mm_cvtpd_ps}    & \texttt{_mm_cvtepi32_ps} \\
% 			\texttt{__m128d}                   & \texttt{_mm_cvtps_pd}    & \texttimes               & \texttt{_mm_cvtepi32_pd} \\
% 			\texttt{__m128i}                   & \texttt{_mm_cvtps_epi32} & \texttt{_mm_cvtpd_epi32} & \texttimes
% 		\end{tabular}
% 	\end{table}
% 	\begin{itemize}
% 		\item When rounding is necesary, the MXCSR register specifies the rounding mode
% 		      \begin{itemize}
% 			      \item [\triangleright] See \texttt{_MM_SET_ROUNDING_MODE}
% 		      \end{itemize}
% 	\end{itemize}
% \end{frame}

\begin{frame}[fragile]
	\frametitle{Loading from Memory}
	\talknote{Aligned ist schneller}
	\talknote{Single lane loads}
	\talknote{Broadcast loads}
	\talknote{Wenn alles entsprechend ausgerichtet: dereferenzierung}
	\begin{columns}[c]
		\begin{column}{0.35\textwidth}
			We can load \ldots
			\begin{itemize}
				\item four values aligned
				\item four values \tikz\node[inline] (U) {unaligned};
				\item four values in reverse
				\item \ldots
			\end{itemize}
		\end{column}
		\begin{column}{0.55\textwidth}
			\begin{visibleenv}<2->
				\begin{minted}[escapeinside=||]{c}
					void mul4(float* vec) {
						__m128 f = |\tikz\node[inline](C){\_mm\_loadu\_ps(vec)};|;
						f = _mm_mul_ps(f, f);
						_mm_storeu_ps(vec, f);
					}
				\end{minted}
			\end{visibleenv}
		\end{column}
	\end{columns}
	\begin{tikzpicture}[remember picture, overlay]
		\draw [textarrow, visible on=<3>] (U) to [bend left=30] (C);
	\end{tikzpicture}
\end{frame}

\begin{frame}[fragile]
	\frametitle{Arithmetic Operations}
	\framesubtitle{For floats (and doubles)}
	\talknote{ss: single lane, ps: alles}
	\talknote{sd, pd: 64 bit float}
	\talknote{C++ min max in SSE implementiert}
	\talknote{[a+b, c-d, e+f, g-h]}
	\talknote{Fused multiply add: (a · b)+c. Genutzt für FLOPs weil zwei operationen}
	\centering
	\begin{columns}[c]
		\begin{column}{0.50\textwidth}
			\texttt{_mm_\fcolorbox{blue}{blue!5}{\phantom{sss}}_\fcolorbox{red}{red!5}{\phantom{ss}}}
			\begin{visibleenv}<3->
				$ = \begin{cases}
						\texttt{_mm_add_ps}                         \\
						\tikz\node[inline](U){\texttt{_mm_mul_ps}}; \\
						\texttt{_mm_min_ss}                         \\
						\ldots
					\end{cases}$
			\end{visibleenv}
		\end{column}
		\begin{column}{0.50\textwidth}
			\begin{visibleenv}<4->
				\begin{minted}[escapeinside=||]{c}
					void mul4(float* vec) {
						__m128 f = _mm_loadu_ps(vec);
						f = |\tikz\node[inline](C){\_mm\_mul\_ps(f, f)};|;
						_mm_storeu_ps(vec, f);
					}
				\end{minted}
			\end{visibleenv}
		\end{column}
	\end{columns}
	\begin{visibleenv}<2->
		\begin{columns}[c]
			\begin{column}{0.35\textwidth}
				\begin{beamerbox}{blue}{}
					\begin{columns}[T]
						\begin{column}{0.45\textwidth}
							\begin{itemize}
								\item add
								\item sub
								\item mul
								\item div
							\end{itemize}
						\end{column}
						\begin{column}{0.45\textwidth}
							\begin{itemize}
								\item sqrt
								\item min
								\item max
								\item \ldots
							\end{itemize}
						\end{column}
					\end{columns}
				\end{beamerbox}
			\end{column}
			\begin{column}{0.35\textwidth}
				\begin{beamerbox}{red}{}
					\begin{columns}[T]
						\begin{column}{0.45\textwidth}
							\begin{itemize}
								\item ss
								\item ps
							\end{itemize}
						\end{column}
						\begin{column}{0.45\textwidth}
							\begin{itemize}
								\item sd
								\item pd
							\end{itemize}
						\end{column}
					\end{columns}
				\end{beamerbox}
			\end{column}
		\end{columns}
	\end{visibleenv}
	\begin{tikzpicture}[remember picture, overlay]
		\draw [textarrow, visible on=<5->] (U.east) to [bend left=20] (C);
	\end{tikzpicture}
\end{frame}

\begin{frame}
	\frametitle{Arithmetic Operations}
	\framesubtitle{For integers}
	\talknote{adds: Sättigungs arithmetik. Overflow, oder max oder min}
	Well, it's the same. Just append \texttt{_epi8} or \texttt{_epi16}
	\talknote{16 chars: 100+100=127}
	\talknote{kein div}
	\talknote{mul [a,b,c,d] [e,f,g,h] = [a*e, c*g]}
	\talknote{besser SSE4.1}
	\begin{center}%
		\texttt{_mm_\fcolorbox{blue}{blue!5}{\phantom{sss}}_\fcolorbox{red}{red!5}{\phantom{ss}}}
	\end{center}
	\begin{columns}[c]
		\begin{column}{0.35\textwidth}
			\begin{beamerbox}{blue}{}
				\begin{columns}[T]
					\begin{column}{0.45\textwidth}
						\begin{itemize}
							\item add
							\item sub
							\item mul
							\item avg
						\end{itemize}
					\end{column}
					\begin{column}{0.45\textwidth}
						\begin{itemize}
							\item min
							\item max
							\item \ldots
						\end{itemize}
					\end{column}
				\end{columns}
			\end{beamerbox}
		\end{column}
		\begin{column}{0.35\textwidth}
			\begin{beamerbox}{red}{}
				\begin{itemize}
					\item epi8
					\item epi16
					\item epu8
				\end{itemize}
			\end{beamerbox}
		\end{column}
	\end{columns}
\end{frame}

\begin{frame}[fragile]
	\frametitle{Storing to Memory}
	\talknote{higher lower}
	\begin{columns}[c]
		\begin{column}{0.35\textwidth}
			We can store \ldots
			\begin{itemize}
				\item four values aligned
				\item four values \tikz\node[inline] (U) {unaligned};
				\item four values in reverse
				\item \ldots
			\end{itemize}
		\end{column}
		\begin{column}{0.55\textwidth}
			\begin{visibleenv}<2->
				\begin{minted}[escapeinside=||]{c}
					void mul4(float* vec) {
						__m128 f = _mm_loadu_ps(vec);
						f = _mm_mul_ps(f, f);
						|\tikz\node[inline](C){\_mm\_storeu\_ps(vec, f)};|;
					}
				\end{minted}
			\end{visibleenv}
		\end{column}
	\end{columns}
	\begin{tikzpicture}[remember picture, overlay]
		\draw [textarrow, visible on=<3>] (U.east) to [bend left=20] (C);
	\end{tikzpicture}
\end{frame}

\begin{frame}
	\frametitle{Miscellaneous}
	\talknote{_may_i_use_cpu_feature Funktion}
	\begin{columns}[c]
		\begin{column}{0.45\textwidth}
			\begin{visibleenv}<2->
				\begin{block}[Copy values to general purpose registers]
					\begin{description}
						\item [\mintinline{c}{_m128i} $\rightarrow$ \mintinline{c}{int32_t}]
						      \texttt{_mm_cvtsi128_si32}
						\item [\mintinline{c}{int32_t} $\rightarrow$ \mintinline{c}{_m128i}]
						      \texttt{_mm_cvtsi32_si128}
						\item [\mintinline{c}{_m128} $\rightarrow$ \mintinline{c}{float}]
						      \texttt{_mm_cvtss_f32}
					\end{description}
				\end{block}
			\end{visibleenv}
		\end{column}
		\begin{column}{0.45\textwidth}
			\begin{visibleenv}<3->
				\begin{block}[Cryptography]
					\begin{itemize}
						\item AES de- and encryption
						\item SHA computation
					\end{itemize}
				\end{block}
			\end{visibleenv}
			\begin{visibleenv}<4->
				\begin{block}[String manipulation (SSE 4.2)]
					\begin{itemize}
						\item Compare strings with \ldots
						      \begin{itemize}
							      \item known length
							      \item unknown length
						      \end{itemize}
					\end{itemize}
				\end{block}
			\end{visibleenv}
		\end{column}
	\end{columns}
\end{frame}

\subsection{Example}
\begin{frame}[fragile]
	\frametitle[\href{https://godbolt.org/z/T3aPb6Trj}{Click for code}]{More adding}
	\talknote{Beliebig viele floats paarweise addieren}
	\pdfpcnote{1: die Anzahl an vektorisierbaren elementen bestimmen}
	\talknote<6>{Auch schleifen. Aber wesentlich kürzer < 4}
	\mode<beamer>
	\begin{onlyenv}<+>
		\begin{minted}[fontsize=\footnotesize]{cpp}
			float* add(const float* a, const float* b, size_t size) {
				float* result = new float[size];

				unsigned i = 0;
				for (; i <                            ; i +=  ) {




				}


				return result;
			}
		\end{minted}
	\end{onlyenv}
	\begin{onlyenv}<+>
		\begin{minted}[fontsize=\footnotesize]{cpp}
			float* add(const float* a, const float* b, size_t size) {
				float* result = new float[size];
				const auto numof_vectorizable_elements = size - (size % 4);
				unsigned i = 0;
				for (; i < numof_vectorizable_elements; i += 4) {




				}


				return result;
			}
		\end{minted}
	\end{onlyenv}
	\begin{onlyenv}<+>
		\begin{minted}[fontsize=\footnotesize]{cpp}
			float* add(const float* a, const float* b, size_t size) {
				float* result = new float[size];
				const auto numof_vectorizable_elements = size - (size % 4);
				unsigned i = 0;
				for (; i < numof_vectorizable_elements; i += 4) {
					__m128 a_reg = _mm_loadu_ps(a + i);
					__m128 b_reg = _mm_loadu_ps(b + i);


				}


				return result;
			}
		\end{minted}
	\end{onlyenv}
	\begin{onlyenv}<+>
		\begin{minted}[fontsize=\footnotesize]{cpp}
			float* add(const float* a, const float* b, size_t size) {
				float* result = new float[size];
				const auto numof_vectorizable_elements = size - (size % 4);
				unsigned i = 0;
				for (; i < numof_vectorizable_elements; i += 4) {
					__m128 a_reg = _mm_loadu_ps(a + i);
					__m128 b_reg = _mm_loadu_ps(b + i);
					__m128 sum = _mm_add_ps(a_reg, b_reg);

				}


				return result;
			}
		\end{minted}
	\end{onlyenv}
	\begin{onlyenv}<+>
		\begin{minted}[fontsize=\footnotesize]{cpp}
			float* add(const float* a, const float* b, size_t size) {
				float* result = new float[size];
				const auto numof_vectorizable_elements = size - (size % 4);
				unsigned i = 0;
				for (; i < numof_vectorizable_elements; i += 4) {
					__m128 a_reg = _mm_loadu_ps(a + i);
					__m128 b_reg = _mm_loadu_ps(b + i);
					__m128 sum = _mm_add_ps(a_reg, b_reg);
					_mm_storeu_ps(result + i, sum);
				}


				return result;
			}
		\end{minted}
	\end{onlyenv}
	\mode<all>
	\begin{onlyenv}<+>
		\begin{minted}[fontsize=\footnotesize]{cpp}
			float* add(const float* a, const float* b, size_t size) {
				float* result = new float[size];
				const auto numof_vectorizable_elements = size - (size % 4);
				unsigned i = 0;
				for (; i < numof_vectorizable_elements; i += 4) {
					__m128 a_reg = _mm_loadu_ps(a + i);
					__m128 b_reg = _mm_loadu_ps(b + i);
					__m128 sum = _mm_add_ps(a_reg, b_reg);
					_mm_storeu_ps(result + i, sum);
				}
				for (; i < size; ++i)
					result[i] = a[i] + b[i];
				return result;
			}
		\end{minted}
	\end{onlyenv}
\end{frame}

\section{Summary}
\begin{frame}
	\frametitle{Should you even care?}
	\begin{columns}[c]
		\begin{column}{0.65\textwidth}
			You shouldn't if \ldots
			\begin{itemize}
				\item you don't write performance sensitive code
				\item you're code is not CPU bound
				\item if most of the math you do is implemented in libraries
				\item if your favourite language does not support SIMD
			\end{itemize}
		\end{column}
		\begin{column}{0.25\textwidth}
			\only<2>{\begin{tikzpicture}
					\chicken[body=accent, graduate=black, tassel=red, speech={\small Don't}]
				\end{tikzpicture}}
		\end{column}
	\end{columns}
\end{frame}

\begin{frame}
	\frametitle{What I didn't cover}
	\begin{columns}[c]
		\begin{column}{0.45\textwidth}
			\begin{visibleenv}<2->
				\begin{block}[Instructions]
					\begin{itemize}
						\item Casting
						\item Converting
						\item Comparing
						\item Shuffling
						\item Shifting
						\item Logic operations
						\item Bitwise operations
						\item Prefetching
					\end{itemize}
				\end{block}
			\end{visibleenv}
		\end{column}
		\begin{column}{0.45\textwidth}
			\begin{visibleenv}<3->
				\begin{block}[Libraries]
					\begin{itemize}
						\item std::experimental::simd
						\item Eigen
						\item DirectXMath
					\end{itemize}
				\end{block}
			\end{visibleenv}
			\begin{visibleenv}<4->
				\begin{beamerbox}{green}{}
					\centering
					\begin{tikzpicture}
						\koala[speech={A lot}, umbrellaclosed=blue]
					\end{tikzpicture}
				\end{beamerbox}
			\end{visibleenv}
		\end{column}
	\end{columns}
\end{frame}

\section{References}
\begin{frame}[allowframebreaks]
	\frametitle{References}
	\printbibliography
\end{frame}
\end{document}